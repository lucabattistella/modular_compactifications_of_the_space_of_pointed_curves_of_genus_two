% user docs for compositio.cls
\def\Filedate{2011/01/24} % date this file last revised
\def\Fileversion{1.15}    % version of compositio.cls documented

\documentclass[noams]{compositio}
% If you have the AMSLaTeX distribution installed on your system,
% please delete the "[noams]" option above.

% definitions specific to this author guide only
\newcommand*{\code}[1]{{\mdseries\texttt{#1}}}
\newcommand*{\pkg}[1]{{\mdseries\textsf{#1}}}
\renewcommand{\topfraction}{0.99}
\renewcommand{\contentsname}{Contents\\{\footnotesize\normalfont(A table
of contents should normally not be included)}}
%
\begin{document}

\title{Using the \pkg{compositio} class file}
%
\author{Fran Burstall}
\email{}
\address{}
%
\author{Ola T\"{o}rnkvist}
\email{compmath@lms.ac.uk}
\address{}
%
\dedication{A dedication can be included here.}
\classification{35J25 (primary), 28C15, 28D10 (secondary). At
least one subject code is required. Please refer to
\url{http://www.ams.org/msc/} for a list of codes.}
\keywords{Please provide keywords here.}
\thanks{This file documents \pkg{compositio} version \Fileversion\ and
was last revised \Filedate.}

\begin{abstract}
The \pkg{compositio} \LaTeX\ document class is designed to give
the page layout, front matter and formatting required for articles
published in \emph{Compositio Mathematica}. The published version
of your article will display the Compositio Mathematica
\textit{logo} and bibliographic information in the upper left-hand
corner of the title page (see one of the sample files
\code{cmguide1.pdf} or \code{cmguide1.ps} available at
\url{http://www.compositio.nl/cmauthor.html}).

An English abstract of less than 200 words is required and should
contain at least two sentences. Please do not include citations,
footnotes or references to numbered equations, theorems, figures
or tables in your abstract. Avoid complicated formulae or
displayed equations, if possible.
\end{abstract}

\maketitle

\vspace*{6pt}\tableofcontents  % for this guide only.
% A table of contents should normally not be included

\section{Introduction}
\label{sec:introduction}

The \pkg{compositio} \LaTeX\ document class is based on the
standard \pkg{article} class that you already know and love. While
\pkg{compositio} performs most of its work behind the scenes, it
does provide some user-visible features:
\begin{itemize}
\item Author ``front matter'' commands are available to provide
authors' email and spatial addresses, dedications, keywords and MSC
classifications (see Section~\ref{sec:front-matter}).
\item Where available, the \pkg{amsthm} and \pkg{amssymb} packages
are loaded to provide rich support for \texttt{theorem}-like
environments and access to the AMS fonts such as blackboard bold
and fraktur (see Subsection \ref{sec:maths}).
\item Extra environments for proofs (with optional end-of-proof
boxes) and acknowledgements are available.
\end{itemize}

This document describes these additional features and outlines the
structure your electronic manuscript should have.

\section{Getting started}
\label{sec:getting-started}
First things first: your document should load the \pkg{compositio}
class and so begin:
\begin{verbatim}
\documentclass{compositio}
\end{verbatim}
Thereafter, have a look at the sample shown in
Figure~\ref{fig:ex1} on page~\pageref{fig:ex1}.  This tells most
of the story. You will find the details in the subsequent sections.
\begin{figure}[!t]\vspace*{6pt}
\hspace*{1in}\begin{minipage}{5.1in}\small\begin{verbatim}
\documentclass{compositio}     % 'we are using the compositio class'

%% packages
\usepackage{amsmath}            % 'not essential, but very useful'

\begin{document}

\title
  [Short Title]         % 'optional short form; for the running head'
  {A longish title like this goes here} % 'Main title'

% 'Each author has his or her own set of coordinates.'
\author{F. Author}
\email{f.author@some.where.ac.uk}  % 'optional'
\address{Mathematics Department\\Some University\\Where Road\\%
         Here\\MT55 9XX}
\curraddr{Mathematics Institute\\Another University\\
          There\\BA1 1HZ}          % 'also optional'
% 'Another author'
\author{A. Author}
\email{another.author@other.edu}
\address{Physics Department\\Other University\\There\\USA}
%
\shortauthors{F. Author et al.}
                        % 'optional short form; for the running head'

% 'MSC classification, keywords and grant acknowledgements'
\classification{58E20}
\keywords{harmonic map, projective geometry}
\thanks{The first author is supported by EPSRC.}

% 'Abstract comes before maketitle, as in the AMS classes'
\begin{abstract}
% 'Abstract text'
\end{abstract}

\maketitle

% 'Main text starts here.'
\end{verbatim}\end{minipage}\vspace*{5mm}
\caption{Document example} \label{fig:ex1}
\end{figure}

\section{Front matter}
\label{sec:front-matter}

Front matter commands follow \verb+\begin{document}+ and precede
\verb+\maketitle+.  They come in two flavours: those that give the
coordinates of authors and those that provide extra information
about the text (keywords and MSC subject classification).

Each author is named in a separate \verb+\author+ command (the
\verb+\and+ mechanism of the \pkg{article} class is not available
here -- see Appendix~\ref{sec:conv-from-textsf}).  After this, use
\verb+\email+ (optional), \verb+\address+ and \verb+\curraddr+
(also optional) \emph{in this order} to indicate the author's
email address, permanent address and, if necessary, temporary
address. Look at Figure~\ref{fig:ex1} for examples. For
compatibility with the \pkg{amsart} class, these commands accept
an optional argument but this is silently ignored.

The list of authors is combined to make the running header on the top
of even-numbered pages.  If there are too many authors to fit,
provide a short alternative with \verb+\shortauthors+:

\verb+\shortauthors{F.E. Burstall et al.}+

\noindent
This will only effect running headers, not the title page.

Use \verb+\classification+ or \verb+\subjclass+ to record the MSC
subject classification, \verb+\keywords+ for (guess what?) listing
keywords describing your text and \verb+\thanks+ to acknowledge
financial and institutional support.  Thus:
\begin{verbatim}
\classification{58E20}
\keywords{harmonic map, projective geometry}
\thanks{The first author is supported by EPSRC.}
\end{verbatim}
You may also insert a dedication with the \verb+\dedication+
command.

Of the front-matter commands, only \verb+\author+, \verb+\address+,
\verb+\classification+ are considered to be compulsory. A warning
will be generated on the screen if \verb+\address+ or
\verb+\classification+ is missing (unless the \texttt{draft} option
is set; see Section~\ref{sec:options}).

\section{Environments}
\label{sec:Environments}

\subsection{Mathematics}
\label{sec:maths}

\pkg{compositio} uses the \textsf{amsthm} package to provide
enhanced support for theorem-like environments.  This introduces
three new features:  the \verb+\theoremstyle+ command for making
different kinds of theorem-like environment, a starred form
\verb+\newtheorem*+ for un-numbered environments of this kind and
a \texttt{proof} environment.

\textsf{amsthm} defines three styles:
  \begin{description}
  \item[plain] The ``usual'' style for a \textsc{Theorem},
  \textsc{Lemma}, \textsc{Corollary} or \textsc{Proposition}.
  \textit{This is the default, with the body typeset in italic
  font.}
  \item[definition] For an \textsc{Assertion}, a \textsc{Conjecture},
  \textsc{Definition} or \textsc{Hypothesis}:
  the body of the statement is in normal, upright font.
  \item[remark] For a \emph{Remark}, \emph{Note}, \emph{Notation},
  an \emph{Observation}, a \emph{Problem}, \emph{Question},
  \emph{Algorithm} or \emph{Example}: here the caption is set in
  italics and the body in an upright font.
 \end{description}
See Figure~\ref{fig:2} for the sort of thing you should put in the
preamble.  You can find more details in the \pkg{amsthm}
documentation\footnote{At
\url{ftp://ftp.tex.ac.uk/tex-archive/macros/latex/required/amslatex/classes/amsthdoc.tex},
for \mbox{example}.}.
\begin{figure}[htbp]
\begin{center}
\begin{minipage}{15cm}\small\begin{verbatim}
% \theoremstyle{plain} % 'this is the initial setting and can be omitted here'
\newtheorem{thm}{Theorem}[section] % number like 3.1, 3.2, 3.3, etc.
\newtheorem*{FLT}{Fermat's Last Theorem} % not numbered
\theoremstyle{definition} % 'here we change the style'
\newtheorem{defn}[thm]{Definition} % numbered with thm
\theoremstyle{remark} % 'style changed again'
\newtheorem{rem}{Remark}
\newtheorem{conj}{Conjecture}
\end{verbatim}
\end{minipage}
\end{center}
\caption{Defining theorem-like environments.} \label{fig:2}
\end{figure}

The environments defined by \verb+\newtheorem+ take an optional
argument often used to indicate an attribution.  Please note that,
in contrast to the \pkg{article} and \pkg{amsart} classes,
\pkg{compositio} does \emph{not} surround the argument with
parentheses: this is to facilitate attributions consisting of a
single reference to the bibliography.  If you want the parentheses,
you must include them in the argument yourself as shown here:

\verb+\begin{thm}[(Burstall, 2002)]+

\noindent
If your site does not have the AMS\LaTeX\ distribution, you should
follow these steps:
\begin{enumerate}
\item Complain to your System Administrator!  The AMS\LaTeX\
distribution is a required component of a well-founded \LaTeX\
installation.
\item Use the \texttt{noams} option to work around the missing files;
see Subsection~\ref{sec:docum-class-opti} for details.
\end{enumerate}

\subsection{Proofs}
\label{sec:proofs}

\pkg{compositio} also provides a \texttt{proof} environment adapted
from that of \pkg{amsthm}.  This takes an  optional argument that
\textit{replaces} the label.  Thus

\verb+         \begin{proof}[Proof of Main Theorem]+

\noindent begins a proof headed \emph{Proof of Main Theorem} rather
than \emph{Proof}.

By default, the proofs are ended with ``Halmos tombstones'' --
open-face boxes. The generation of these boxes can be turned off
everywhere by use of the \texttt{noboxes} option; see
Section~\ref{sec:docum-class-opti}.

\subsection{Acknowledgements}
\label{sec:acknowledgements}

Use the \texttt{acknowledgements} environment to wrap your
thanks to colleagues for their inspirational conversations.

\section{Options}
\label{sec:options}

\subsection{Document class options}
\label{sec:docum-class-opti}

\pkg{compositio} is designed especially for \emph{Compositio
Mathematica} and most font-size and layout decisions have already
been made. Therefore, most of the options of the \pkg{article}
class are inappropriate.  To help transfer documents from
\pkg{article} to \pkg{compositio}, the following document class
options are accepted (perhaps with a warning), but do not affect
the document.
\begin{verbatim}
10pt 11pt 12pt
oneside twoside
onecolumn twocolumn
a4paper
letterpaper
\end{verbatim}
That said, \pkg{compositio} has a few options of its own:
\begin{description}
\item[draft] This suppresses warnings about elements of front
matter. It also marks over-long lines with ugly black boxes.
\item[noboxes] Suppresses the placement of boxes at the end of
proofs.
\item[noams] If you have not got the AMS packages installed,
then you can use this option. It suppresses the loading
of the packages and defines \verb+\theoremstyle+ and
\verb+\newtheorem*+ to be silently accepted without error.
Although the theorem commands will not work as intended in this
case, authors should still use them, so that the final form of the
document generated by \emph{Compositio Mathematica} will use the
AMS definitions where appropriate. Similarly, authors should still
use \verb+\mathbb+ to specify blackboard bold, although the
\code{noams} option defines this command to produce normal bold.
\end{description}

\subsection{The \code{compositio.cfg} configuration file}
\label{sec:mdser-conf-file}

If a file with the name \texttt{compositio.cfg} is present in the
\LaTeX\ search path, it will be input by \pkg{compositio} before it
processes the options. In this case, \emph{all options specified in
the document will be ignored}.

\section{Graphics and tables}
\label{sec:graphics-tables}

\pkg{compositio} pre-loads the standard \pkg{graphicx} package, which
allows easy loading of PostScript images generated by external
programs. Figure~\ref{fig:3} gives the idea: here the image is scaled
to have a width of 3 inches.  Note the use of \verb+\centering+
rather than a \texttt{center} environment; this prevents the
appearance of extraneous white space.
\begin{figure}[hbpt]
\centering
\begin{minipage}{3in}
\small\begin{verbatim}
\begin{figure}
\centering
\includegraphics[width=3in]{image.ps}
\caption{A Klein bottle}
\end{figure}
\end{verbatim}
\caption{Including an image} \label{fig:3}
\end{minipage}
\end{figure}

The \pkg{compositio} house style prescribes that the caption for
a figure should follow the image while the caption for a table
should precede it. To achieve this effect, simply place the
\verb+\caption+ command after the \verb+\includegraphics+ command
in a figure but \emph{before} the \texttt{tabular} environment in
a \texttt{table} environment.

\section{Bibliography}\label{refs} A bibliography suitable for
\emph{Compositio Mathematica} can be created with \textsc{Bib}\TeX\
using the AMS bibliography style \pkg{amsalpha}. The
\textsc{References} section is then generated with the commands
\newline\hspace*{1.5em}%
\verb+\bibliographystyle{amsalpha}+\newline\hspace*{1.5em}%
\verb+\bibliography{+\textit{bibname}\verb+}+\newline where
\textit{bibname} is the first part of the name of a
\textsc{Bib}\TeX\ file with extension \code{.bib}.

If the \textsc{References} section is entered by hand, please use
the first [optional] argument of the \verb+\bibitem+ command to
provide acronyms formed by the first letters of the authors'
surnames and the last two digits of the publication year (followed
by ``a'',``b'', etc.\ when needed). The bibliography for this
guide was generated using the following lines of code:
\begin{center}\begin{minipage}{5in} {\footnotesize%
\begin{verbatim}
\begin{thebibliography}{PTW02} % '2nd argument contains the widest acronym'
\bibitem[Lam94]{Lamport}
  L. Lamport, \emph{\LaTeX: A document preparation system \textup{(}%
  updated for \LaTeXe\textup{)}} (Addison--Wesley, New York, 1994).
\bibitem[PTW02]{PRL}
  T.~Prokopec, O.~T\"ornkvist and R.~P.~Woodard, \emph{Photon mass
  from inflation}, Phys.\ Rev.\ Lett.\ \textbf{89} (2002), 101301.
\end{thebibliography}
\end{verbatim}}
\end{minipage}\end{center}
The publications are cited using the second argument of
\verb+\bibitem+, \textit{e.g.}\
\verb+\cite{Lamport}+~\cite{Lamport}.

\appendix
\section{Converting from other document classes}
\label{sec:conv-from-textsf}

The \pkg{compositio} class has been designed to make conversion
from other popular document classes as easy as possible.  The
first thing to do is to change the documentclass.  Thus, replace
\texttt{article} or \texttt{amsart} in the \verb+\documentclass+
command to give \verb+\documentclass{compositio}+.

\subsection{Converting from \pkg{article}}
\label{sec:conv-from-pkgart}

The main differences between \pkg{article} and \pkg{compositio} lie
in the treatment of front matter.  With \pkg{article}, the
\verb+\author+ command had to do all the work of displaying authors'
names, addresses and acknowledgements, whereas \pkg{compositio} has
separate commands for all these.  Moreover, the \pkg{compositio}
class does not support the \verb+\and+ method of fitting several
authors into a single \verb+\author+ command.  This is because
addresses are provided for each author individually.  Thus you must
replace
\begin{verbatim}
\author{H.~Jones \and J.~Smith} % wrong!
\end{verbatim}
with
\begin{verbatim}
\author{H.~Jones}
\author{J.~Smith}
\end{verbatim}
Furthermore, you need to remove all address lines from
\verb+\author+ and place them in \verb+\address+ commands, one for
each author. Similarly, \verb+\thanks+ should be taken out of
\verb+\author+ and placed by itself.  To summarise: if your
document has something like this
\begin{verbatim}
\author{F.E. Burstall\thanks{Supported by EPSRC.}\\
University Of Bath\\ Bath BA1 1HZ\\ UK}
\end{verbatim}
you should edit to achieve this:
\begin{verbatim}
\author{F.E. Burstall}
\address{University Of Bath\\ Bath BA1 1HZ\\ UK}
\thanks{The first author is supported by EPSRC.}
\end{verbatim}
The \pkg{compositio} version of \verb+\thanks+ does not decorate the
authors with footnote symbols to indicate the mapping between
different authors and their support.  Thus, with multiple authors,
you should use one \verb+\thanks+
of the form \verb+\thanks{The first author+%
\ldots\verb+The second author+\ldots\verb+}+.

Having successfully converted \verb+\author+, you may want to add
email addresses with \verb+\email+ (as well as MSC subject
classifications and keywords). To see how to do this, see
Section~\ref{sec:front-matter}.

\subsection{Converting from \pkg{amsart}}
\label{sec:conv-from-pkgams}

The \pkg{compositio} class supports the same address commands as
the \pkg{amsart} class, but stylistic considerations force a few
differences:
\begin{enumerate}
\item The \verb+\email+ command comes between \verb+\author+ and
\verb+\address+ rather than after \verb+\address+; \item each
\verb+\author+ \emph{must} have his or her own \verb+\address+
even if two authors share an address; \item all acknowledgements
of support should be collected into a single \verb+\thanks+.
\end{enumerate}
As a consequence of the second item, any optional arguments to the
AMS versions of \verb+\email+, \verb+\address+ and
\verb+\curraddr+ are silently ignored.

The only other issue is that \pkg{amsart} automatically loads the
\pkg{amsmath} package.  Consequently, you should place a
\verb+\usepackage{amsmath}+ in the preamble of your converted
document.

\subsection{Converting from \LaTeX2.09}
\label{sec:conv-from-latex209}

If your document begins \verb+\documentstyle+, then you are using
a dialect of \LaTeX\ that has been obsolete since 1994 and you are
in trouble. In the case that your \LaTeX\ installation really is
that old, you can do nothing but complain to your Systems
Administrator; \pkg{compositio} will not work for you. Otherwise,
first replace \verb+\documentstyle+ with \verb+\documentclass+, then
follow the instructions in Section~\ref{sec:conv-from-pkgart} and
try running \LaTeX\ on the converted document.

If you run into errors, try removing any optional arguments to
\verb+\documentclass+ and then seek the advice of your local \LaTeX\
guru.

\begin{acknowledgements}
Parts of this document were ruthlessly plagiarised from David
Carlisle's \texttt{jcmguide.tex}.  The authors gratefully confess
this theft here.
\end{acknowledgements}

\begin{thebibliography}{PTW02} % '2nd argument contains the widest acronym'
\bibitem[Lam94]{Lamport}
  L.~Lamport, \emph{\LaTeX: A document preparation system \textup{(}%
  updated for \LaTeXe\textup{)}} (Addison--Wesley, New York, 1994).
\bibitem[PTW02]{PRL}
  T.~Prokopec, O.~T\"ornkvist and R.~P.~Woodard, \emph{Photon mass
  from inflation}, Phys.\ Rev.\ Lett.\ \textbf{89} (2002), 101301.
\end{thebibliography}

\end{document}
